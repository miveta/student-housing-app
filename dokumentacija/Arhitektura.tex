\chapter{Arhitektura i dizajn sustava}
		
			Arhitektura sustava je web aplikacija kojoj će korisnici pristupati pomoću web preglednika. Odlučili smo se na takvu arhitekturu jer je cilj sustava da bude što jednostavniji za korištenje i da mu se može pristupiti sa svih mjesta.
			
			\begin{figure}[H]
				\includegraphics[scale=0.4]{slike/Skica_sustava.png} %veličina slike u odnosu na originalnu datoteku i pozicija slike
				\centering
				\caption{Skica sustava}
				\label{fig:sustav}
			\end{figure}
			
			Programski jezik koji smo odabralni za izradu aplikacije je java s razvojnim okvirom Spring Boot te javaScript. Odabrano razvojno okruženje je IntelliJ IDEA. 
			Aplikacija je organizirana u dva sloja: frontend i backend. Za izradu frontenda koristi se React.React je javaScript biblioteka koja služi za izradu jednostranične aplikacije. Frontend i backend komuniciraju pomoću RESTa. REST se bazira na HTTP protokolu. Backend se sastoji od pet komponenti:
		\begin{itemize}
		\item Kontroler - služi za komunikaciju s frontendom. Zaprima HTTP zathtjev te određuje koja će se funkcionalnost izvršavati
		\item Servis - u njima se odvijaju poslovne logike i sve funkcionalnosti aplikacije
		\item Repozitorij - dohvaća i sprema podatke u bazu podataka
		\item Model - opisuju entitete iz baze 
		\item Security - omogućava autentikaciju i autorizaciju 
	\end{itemize}

\begin{figure}[H]
\includegraphics[scale=0.4]{slike/Skica_aplikacije.png} %veličina slike u odnosu na originalnu datoteku i pozicija slike
\centering
\caption{Skica aplikacije}
\label{fig:aplikacija}
\end{figure}


	
		

		

				
		\section{Baza podataka}
			
		Za potrebe sustava za zamjenu soba koristit ćemo relacijsku bazu podataka koja nam omogućuje oblikovanje objekata iz stvarnog svijeta pomoću povezanih tablica - relacija. Svaka je tablica definirana vlastitim nazivom i skupom različitih atributa koji je opisuju. Glavna je zadaća baze podataka pohrana, brzo pronalaženje i dohvaćanje te dodavanje i brisanje podataka. Baza podataka ovog sustava sastoji se od entiteta:
		\begin{itemize}
			\item Student
			\item Oglas
			\item Soba
			\item Grad
			\item Dom
			\item Paviljon
			\item StudentskiCentar
			\item Obavijest
			\item Zaposlenik SC
			\item TrazeniUvjeti
			\item Lajkovi
			\item StudentObavijesti
			
		\end{itemize}
		
			\subsection{Opis tablica}
			

			\textbf{Student } Entitet sadrži informacije o korisniku aplikacije - studentu. Sadrži sljedeće atribute: identifikator studenta, korisničko ime, ime, prezime, e-mail adresu, JMBAG, lozinku te oznaku za primanje mailova. Entitet je u vezi \textit{Many-to-Many} s entitetom Obavijest preko identifikatora obavijesti, u vezi \textit{One-to-One} s entitetom Oglas preko atributa statusa oglasa, u vezi \textit{One-to-One} s entitetom TrazeniUvjeti preko atributa identifikatora uvjeta te u vezi \textit{One-to-One} s entitetom Oglas. 
				
				
				
				\begin{longtabu} to \textwidth {|X[6, 2]|X[6, 2]|X[20, l]|}
					
					\hline \multicolumn{3}{|c|}{\textbf{Student}}	 \\[3pt] \hline
					\endfirsthead
					
					\hline \multicolumn{3}{|c|}{\textbf{Student}}	 \\[3pt] \hline
					\endhead
					
					\hline 
					\endlastfoot
					
					\textbf{idKorisnik} & UUID	& jedinstveni identifikator studenta (korisnika) 	\\ \hline
					korisnickoIme	& VARCHAR & jedinstveno korisničko ime  	\\ \hline 
					jmbag & VARCHAR & jedinstveni JMBAG studenta \\ \hline 
					ime & VARCHAR & ime studenta 		\\ \hline
					prezime & VARCHAR & prezime studenta \\ \hline
					email & VARCHAR & e-mail adresa studenta \\ \hline
					hashLozinke & VARCHAR & hash lozinka \\ \hline
					obavijestiNaMail & BOOLEAN & oznaka želi li student primati obavijesti na mail \\ \hline
					viseOvlasti & BOOLEAN & aa \\ \hline
					\textit{idStatusOglasa} & BOOLEAN & oznaka potvrde \\ \hline
					\textit{idTrazeniUvjeti} & VARCHAR & traženi kriteriji za sobu za zamjenu \\ \hline
				 
					
				\end{longtabu}
			
				\textbf{Oglas } Entitet sadrži informacije koje su vezane uz oglas koji student predaje. Sadrži atribute: ID oglasa, naslov oglasa, opis te datum objave oglasa. Entitet je u vezi \textit{One-to-One} s entitetom Status preko atributa identifiaktora statusa oglasa i u vezi \textit{One-to-Many} s entitetom Obavijest preko atributa identifikatora oglasa.
			
				\begin{longtabu} to \textwidth {|X[6, 2]|X[6, 2]|X[20, l]|}
					
					\hline \multicolumn{3}{|c|}{\textbf{Oglas}}	 \\[3pt] \hline
					\endfirsthead
					
					\hline \multicolumn{3}{|c|}{\textbf{Oglas}}	 \\[3pt] \hline
					\endhead
					
					\hline 
					\endlastfoot
					
					\textbf{idOglas} & UUID	& jedinstveni identifikator oglasa 	\\ \hline
					naslov	& VARCHAR & naslov oglasa  	\\ \hline 
					opis & VARCHAR & opis oglasa \\ \hline 
					objavljen & DATE & datum objavljivanja oglasa 		\\ \hline
					godina & INTEGER & godina objavljivanja oglasa \\ \hline
					\textit{idStatusOglasa} & VARCHAR & status oglasa \\ \hline
					
					 
					
					
				\end{longtabu}
			
				\textbf{Soba } Entitet sadrži informacije o sobi u studentskom domu. Sadrži atribute: broj sobe, kat na kojemu se soba nalazi,broj kreveta i vrstu kupaonice koja pripada sobi, kategoriju te identifikator paviljona i doma. Entitet je u vezi \textit{Many-to-One} s entitetom Paviljon preko atributa identifikatora paviljona i doma.
			
				\begin{longtabu} to \textwidth {|X[6, 2]|X[6, 2]|X[20, l]|}
					
					\hline \multicolumn{3}{|c|}{\textbf{Soba}}	 \\[3pt] \hline
					\endfirsthead
					
					\hline \multicolumn{3}{|c|}{\textbf{Soba}}	 \\[3pt] \hline
					\endhead
					
					\hline 
					\endlastfoot
					
					\textbf{broj} & INTEGER & broj sobe 	\\ \hline
					\textbf{kat} & INTEGER & kat na kojemu se soba nalazi \\ \hline 
					brojKreveta & VARCHAR & broj kreveta u sobi \\ \hline
					tipKupaonice & VARCHAR & vrsta dostupne kupaonice \\ \hline
					kategorija & VARCHAR & kategorija sobe \\ \hline
					\textbf{\textit{idPaviljon}} & UUID & identifikator paviljona kojemu soba pripada \\ \hline
					\textbf{\textit{idDom}} & UUID & identifikator doma kojemu soba pripada \\ \hline
					
					
				\end{longtabu}
			
				\textbf{Grad } Entitet sadrži informacije o pojedinom gradu. Sadrži atribute: identifikator grada i naziv te identifikator studentskog centra tog grada. Entitet je u vezi \textit{One-To-One} s entitetom Studentski Centar preko atributa identifikatora studentskog centra i u vezi \textit{One-to-Many} s entitetom Dom preko identifikatora grada. 
			
				\begin{longtabu} to \textwidth {|X[6, 2]|X[6, 2]|X[20, l]|}
					
					\hline \multicolumn{3}{|c|}{\textbf{Grad}}	 \\[3pt] \hline
					\endfirsthead
					
					\hline \multicolumn{3}{|c|}{\textbf{Grad}}	 \\[3pt] \hline
					\endhead
					
					\hline 
					\endlastfoot
					
					\textbf{idGrad} & UUID	& jedinstveni identifikator grada	\\ \hline
					naziv	& VARCHAR & ime grada  	\\ \hline  
					\textit{idSc} & UUID & identifikator gradskog studentskog centra \\ \hline
					
					
				\end{longtabu}
			
				\textbf{Dom } Entitet sadrži sve važne informacije o pojedinom studentskom domu. Sadrži atribute: ID doma, naziv doma, ID grada u kojemu se dom nalazi te oznaku ima li dom vlastitu menzu. Entitet je u vezi \textit{Many-to-One} s entitetom Grad preko atributa identifikatora grada i u vezi \text{One-To-Many} s entitetom Paviljon preko identifikatora doma. 
			
				\begin{longtabu} to \textwidth {|X[6, 2]|X[6, 2]|X[20, l]|}
					
					\hline \multicolumn{3}{|c|}{\textbf{Dom}}	 \\[3pt] \hline
					\endfirsthead
					
					\hline \multicolumn{3}{|c|}{\textbf{Dom}}	 \\[3pt] \hline
					\endhead
					
					\hline 
					\endlastfoot
					
					\textbf{idDom} & UUID	& jedinstveni identifikator studentskog doma 	\\ \hline
					naziv	& VARCHAR & ime studentskog doma  	\\ \hline 
					imaMenzu & BOOLEAN & oznaka ima li dom vlastitu menzu \\ \hline
					\textit{idGrad} & UUID & identifikator grada u kojemu se dom nalazi \\ \hline
					
					
				\end{longtabu}
			
				\textbf{Paviljon } Entitet sadrži sve informacije o pojedinom paviljonu studentskog doma. Sadrži atribute: identifikator paviljona, naziv te identifikator doma. Entitet je u vezi \textit{Many-to-One} s entitetom Dom preko atributa identifikatora doma te u vezi \textit{One-to-Many} s entitetom Soba preko indetifikatora paviljona. 
			
				\begin{longtabu} to \textwidth {|X[6, 2]|X[6, 2]|X[20, l]|}
					
					\hline \multicolumn{3}{|c|}{\textbf{Paviljon}}	 \\[3pt] \hline
					\endfirsthead
					
					\hline \multicolumn{3}{|c|}{\textbf{Paviljon}}	 \\[3pt] \hline
					\endhead
					
					\hline 
					\endlastfoot
					
					\textbf{idPaviljon} & UUID	& jedinstveni identifikator paviljona	\\ \hline
					naziv & VARCHAR & naziv paviljona  	\\ \hline 
					\textit{idDom} & UUID & identifikator doma u kojemu se paviljon nalazi \\ \hline
					
					
				\end{longtabu}
			
			
				\textbf{Studentski centar } Entitet sadrži informacije o studentskom centru. Sadrži atribute: identifikator studentskog centra, naziv te identifikator grada u kojemu se studentski centar nalazi. Entitet je u vezi \textit{One-to-One} s entitetom Grad preko atributa identifikatora grada te u vezi \textit{One-to-Many} s entitetom Zaposlenik SC preko identifikatora studentskog centra. 
			
				\begin{longtabu} to \textwidth {|X[6, 2]|X[6, 2]|X[20, l]|}
					
					\hline \multicolumn{3}{|c|}{\textbf{Studentski Centar}}	 \\[3pt] \hline
					\endfirsthead
					
					\hline \multicolumn{3}{|c|}{\textbf{Studentski centar}}	 \\[3pt] \hline
					\endhead
					
					\hline 
					\endlastfoot
					
					\textbf{idSc} & UUID & jedinstveni identifikator studentskog centra	\\ \hline
					naziv  & VARCHAR & ime studentskog centra  	\\ \hline
					\textit{idGrad} & UUID & identifikator grada u kojemu se nalazi studentski centar \\ \hline
					
					
				\end{longtabu}
			
				\textbf{Obavijest} Entitet sadrži informacije o obavijestima koje aplikacija šalje studentima. Sadrži entitete: identifikator obavijesti, tekst, oznaku je li obavijest procitana, vrijeme slanja obavijesti te listu studenata kojima se obavijest šalje i identifikator oglasa za koji se obavijest šalje. Entitet je u vezi \textit{Many-to-Many} s entitetom Student te u vezi \textit{Many-to-One} s entitetom Oglas preko identifikatora oglasa. 
			 
				\begin{longtabu} to \textwidth {|X[6, 2]|X[6, 2]|X[20, l]|}
					
					\hline \multicolumn{3}{|c|}{\textbf{Obavijest}}	 \\[3pt] \hline
					\endfirsthead
					
					\hline \multicolumn{3}{|c|}{\textbf{Obavijest}}	 \\[3pt] \hline
					\endhead
					
					\hline 
					\endlastfoot
					
					\textbf{idObavijest} & UUID & jedinstveni identifikator obavijesti	\\ \hline
					tekst  & VARCHAR & tekst obavijesti  	\\ \hline 
					procitana & BOOLEAN & oznaka je li poruka pročitana \\ \hline
					vrijeme & DATE & vrijeme slanja obavijesti \\ \hline
					\textit{idOglas} & UUID & identifikator oglasa za koji se obavijest generira \\ \hline
					
					
				\end{longtabu}
			
				\textbf{Zaposlenik SC } Entitet sadrži informacije o zaposleniku u studentskom centru. Sadrži atribute: identifikator zaposlenika, korisničko ime i lozinku za prijavu u sustav, ime i prezime zaposlenika, e-mail adresu te identifikator studentskog centra u kojem je zaposlen.
			
				\begin{longtabu} to \textwidth {|X[6, 2]|X[6, 2]|X[20, l]|}
					
					\hline \multicolumn{3}{|c|}{\textbf{Zaposlenik SC}}	 \\[3pt] \hline
					\endfirsthead
					
					\hline \multicolumn{3}{|c|}{\textbf{Zaposlenik SC}}	 \\[3pt] \hline
					\endhead
					
					\hline 
					\endlastfoot
					
					\textbf{idZaposlenik} & UUID	& jedinstveni identifikator zaposlenika studentskog centra	\\ \hline
					korisnickoIme & VARCHAR & jedinstveno korisničko ime zaposlenika studentskog centra \\ \hline
					ime & VARCHAR & ime zaposlenika studentskog centra \\ \hline
					prezime & VARCHAR & prezime zaposlenika studentskog centra \\ \hline
					email & VARCHAR & e-mail adresa zaposlenika studentskog centra \\ \hline
					hashLozinke & VARCHAR & hash lozinke \\ \hline
					\textit{idSc} & UUID & identifikator studentskog centra u kojemu je zaposlen \\ \hline
				
					
					
				\end{longtabu}
			
				\begin{longtabu} to \textwidth {|X[6, 2]|X[6, 2]|X[20, l]|}
					
					\hline \multicolumn{3}{|c|}{\textbf{TrazeniUvjeti}}	 \\[3pt] \hline
					\endfirsthead
					
					\hline \multicolumn{3}{|c|}{\textbf{TrazeniUvjeti}}	 \\[3pt] \hline
					\endhead
					
					\hline 
					\endlastfoot
					
					\textbf{idTrazeniUvjeti} & UUID	& jedinstveni identifikator skupa traženih uvjeta	\\ \hline
					brojKreveta & VARCHAR & traženi broj kreveta \\ \hline
					tipKupaonice & VARCHAR & traženi tip kupaonice \\ \hline
					kateogrija & VARCHAR & tražena kategorija \\ \hline
					godina & INTEGER & godina za koju se predaje oglas \\ \hline
					komentar & VARCHAR & dodatni komentari vezani uz tražene uvjete \\ \hline
					
					
					
					
					
				\end{longtabu}
			
				\textbf{Lajkovi} Entitet sadrži informacije vezane uz 'lajkove' oglasa. Sadrži atribute: identifikator oglasa i identifikator studenta te ocjenu. Entitet je u vezi \textit{Many-to-One} s entitetom Student preko identifikatora studenta i u vezi \textit{Many-to-One} s entitetom Oglas preko identifikatora oglasa.
			
				\begin{longtabu} to \textwidth {|X[6, 2]|X[6, 2]|X[20, l]|}
					
					\hline \multicolumn{3}{|c|}{\textbf{Lajkovi}}	 \\[3pt] \hline
					\endfirsthead
					
					\hline \multicolumn{3}{|c|}{\textbf{Lajkovi}}	 \\[3pt] \hline
					\endhead
					
					\hline 
					\endlastfoot
					
					\textbf{idOglas} & UUID & jedinstveni identifikator oglasa koji se ocjenjuje \\ \hline
					\textbf{idStudent} & UUID & jedinstveni identifikator studenta koji je 'dao lajk' \\ \hline
					ocjena & INTEGER & iznos ocjene \\ \hline
					
					
				\end{longtabu}
			
				\begin{longtabu} to \textwidth {|X[6, 2]|X[6, 2]|X[20, l]|}
					
					\hline \multicolumn{3}{|c|}{\textbf{Status oglasa}}	 \\[3pt] \hline
					\endfirsthead
					
					\hline \multicolumn{3}{|c|}{\textbf{Status oglasa}}	 \\[3pt] \hline
					\endhead
					
					\hline 
					\endlastfoot
					
					\textbf{idStautsOglasa} & UUID & jedinstveni identifikator statusa oglasa \\ \hline
					status & INTEGER & aa \\ \hline
					\textit{idOglas} & UUID & identifikator oglasa\\ \hline
					\textit{idStudent} & UUID & identifikator studenta \\ \hline
					
					
					
					
				\end{longtabu}
			
				\begin{longtabu} to \textwidth {|X[6, 2]|X[6, 2]|X[20, l]|}
					
					\hline \multicolumn{3}{|c|}{\textbf{StudentObavijesti}}	 \\[3pt] \hline
					\endfirsthead
					
					\hline \multicolumn{3}{|c|}{\textbf{StudentObavijesti}}	 \\[3pt] \hline
					\endhead
					
					\hline 
					\endlastfoot
					
					\textit{studentiIdKorisnik} & UUID & aa \\ \hline
					\textit{obavijestiIdObavijest} & UUID & aa \\ \hline
					
					
					
					
				\end{longtabu}
			
				\begin{longtabu} to \textwidth {|X[6, 2]|X[6, 2]|X[20, l]|}
					
					\hline \multicolumn{3}{|c|}{\textbf{Broj kreveta}}	 \\[3pt] \hline
					\endfirsthead
					
					\hline \multicolumn{3}{|c|}{\textbf{Broj kreveta}}	 \\[3pt] \hline
					\endhead
					
					\hline 
					\endlastfoot
					
					
					
					
				\end{longtabu}
			
				
			
				\begin{longtabu} to \textwidth {|X[6, 2]|X[6, 2]|X[20, l]|}
					
					\hline \multicolumn{3}{|c|}{\textbf{Tip kupaonice}}	 \\[3pt] \hline
					\endfirsthead
					
					\hline \multicolumn{3}{|c|}{\textbf{Tip kupaonice}}	 \\[3pt] \hline
					\endhead
					
					\hline 
					\endlastfoot
					
					
					
					
				\end{longtabu}
		
		
			
			
			\subsection{Dijagram baze podataka}
				\begin{figure}[H]
					\includegraphics[scale=0.4]{dijagrami/ERdijagram.png} %veličina slike u odnosu na originalnu datoteku i pozicija slike
					\centering
					\caption{ER dijagram baze podataka}
					\label{fig:er}
				\end{figure}
			
			\eject
			
			
		\section{Dijagram razreda}
		
			Na slikama su prikazani razredi backenda. Na slici 4.4. prikazani su razredi Modela. Modeli opisuju entitete u bazi podataka. Na slici 4.5. prikazani su dijelovi Kontroler, Servis i Repozitorij. Kontroler AuthController manipulira s KorisnikDTO što je \textit{Data transfer object}(DTO) kojeg šaljemo na frontend kako bi mogli pamtiti trenutnog korisnika. Varijable KorsnikDTOa su dohvaćene metodama iz razreda Model.
			Kontroleri također pozivaju servise. Servisi obavljaju logiku aplikacije te za to koriste podatke iz razreda Model. Kontroleri i servisi pozivaju Repozitorij. To su sučelja koja nasljeđuju sučelje JpaRepository koje ima ugrađene metode za dohvat i spremanje podataka u bazu. Na slici 4.6. prikazan je dio Security. Razrede iz dijela Security pozivaju kontroleri za autentikaciju podataka.
		
			\begin{figure}[H]
				\includegraphics[scale=0.3]{dijagrami/model.png} %veličina slike u odnosu na originalnu datoteku i pozicija slike
				\centering
				\caption{Dijagrami razreda - dio Model}
				\label{fig:model}
			\end{figure}
		
			\begin{figure}[H]
				\includegraphics[scale=0.3]{dijagrami/controller.png} %veličina slike u odnosu na originalnu datoteku i pozicija slike
				\centering
				\caption{Dijagrami razreda - dio Kontroler, Servis i Repozitorij}
				\label{fig:controller}
			\end{figure}
		
			\begin{figure}[H]
				\includegraphics[scale=0.3]{dijagrami/security.png} %veličina slike u odnosu na originalnu datoteku i pozicija slike
				\centering
				\caption{Dijagrami razreda - dio Security}
				\label{fig:security}
			\end{figure}
		
		