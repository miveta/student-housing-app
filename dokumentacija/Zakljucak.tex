\chapter{Zaključak i budući rad}
		
		Naš projektni zadatak bio je razviti web aplikaciju pod nazivom "Zamjena sobe" koja bi studentima omogućila organiziranu razmjenu soba u studentskim domovima, a djelatnicima u studentskim centrima omogućila izravan pristup zamjenama koje mogu provesti. Tijekom dva ciklusa naš je tim uložio mnogo vremena i truda u ostvarenje projektnog cilja. Rad na projektu bio je ostvaren kroz dvije glavne faze.
		 \\Prva je faza obuhvatila upoznavanje i formiranje razvojnog tima, upoznavanje s dodijeljenim zadatkom te planiranje rada i podjela uloga. U toj su fazi razrađeni funkcionalni i nefunkcionalni zahtjevi koji su, uz tekstualne opise, bili popraćeni i UML dijagramima obrazaca uporabe, sekvencijskim dijagramima, dijagramima razreda te ER dijagramom baze podataka. Uz razradu zahtjeva, definirala se i arhitektura sustava kao i tehnologije koje će se koristiti za samu implementaciju. Ova je faza postavila temelje razvoja i olakšala daljnji rad na našoj aplikaciji.
		 \\Druga je faza obuhvatila implementaciju rješenja, testiranje rješenja i izradu konačne dokumentacije. Dokumentirani su UML dijagrami stanja, aktivnosti, komponenti i razmještaja koji prikazuju ključne dijelove programskog rješenja. Bez obzira na dobro definiranu strukturu projekta u prvoj fazi, implementacija je zahtijevala mnogo truda i konstantne komunikacije između članova tima. 
		 \\Komunikacija se odvijala putem \textit{WhatsAppa} za brže konzultacije i dogovaranje online sastanaka koji su se održavali putem \textit{Teamsa}. U timu je vladala pozitivna atmosfera što je olakšavalo rad.
		 \\Sudjelovanje i rad na projektu pokazalo se kao vrlo poučno i korisno iskustvo. Članovi tima upoznali su se s mnogim novim alatima i okruženjima, ali i s radom u tim s kojim se dosada u ovoj mjeri nisu susreli. Članovi tima zadovoljni su što su uspjeli ostvariti svoj cilj te što su usvojili mnoge vještine koje će im koristiti u daljnjem obrazovanju i budućem radu.
		
		\eject 
